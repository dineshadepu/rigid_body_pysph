\documentclass[preprint,12pt]{elsarticle}
% \documentclass[draft,12pt]{elsarticle}

\usepackage{hyperref}
\usepackage{graphicx}
\usepackage{subcaption}
\usepackage{amssymb}
\usepackage{amsmath}
\usepackage{multirow}
\usepackage{relsize}
\usepackage[utf8]{inputenc}
\usepackage{cleveref}
\usepackage{algorithm}
\usepackage[noend]{algpseudocode}
\usepackage[section]{placeins}
\usepackage{booktabs}
\usepackage{url}

% For the TODOs
\usepackage{xcolor}
\usepackage{xargs}
\usepackage[colorinlistoftodos,textsize=footnotesize]{todonotes}
\newcommand{\todoin}{\todo[inline]}
% from here: https://tex.stackexchange.com/questions/9796/how-to-add-todo-notes
\newcommandx{\unsure}[2][1=]{\todo[linecolor=red,backgroundcolor=red!25,bordercolor=red,#1]{#2}}
\newcommandx{\change}[2][1=]{\todo[linecolor=blue,backgroundcolor=blue!25,bordercolor=blue,#1]{#2}}
\newcommandx{\info}[2][1=]{\todo[linecolor=OliveGreen,backgroundcolor=OliveGreen!25,bordercolor=OliveGreen,#1]{#2}}

%Boldtype for greek symbols
\newcommand{\teng}[1]{\ensuremath{\boldsymbol{#1}}}
\newcommand{\ten}[1]{\ensuremath{\mathbf{#1}}}

\usepackage{lineno}
% \linenumbers

\journal{}

\begin{document}

\begin{frontmatter}

  \title{}
  \author[XXX]{Dinesh Adepu\corref{cor1}}
  \ead{d.dinesh@surrey.ac.uk}
  \author[XXX]{Pawan Negi \corref{cor1}}
  \ead{xxx}
  \author[University of Surrey]{Chuan Yu Wu}
  \ead{XXX}
\address[xxx]{xxx}

\cortext[cor1]{Corresponding author}


\begin{abstract}
\end{abstract}

\begin{keyword}
%% keywords here, in the form: keyword \sep keyword
{xxx}, {xxx}, {xxx}

%% MSC codes here, in the form: \MSC code \sep code
%% or \MSC[2008] code \sep code (2000 is the default)

\end{keyword}

\end{frontmatter}

% \linenumbers


\FloatBarrier%
\section{Introduction}
\label{sec:conclusions}
Lethe DEM has given benchmarks required to validate our DEM solver.



\FloatBarrier%
\section{Rigid body dynamics 2D}
\label{sec:rb_2d}

% Taken from the appendix of \cite{dietemann2020smoothed}
Linear velocity and center of mass is updated as


\begin{equation}
  \label{eq:rfc:lin_vel_cm_update}
  \ten{v}_{cm}^{n+1} = \ten{v}_{cm}^{n} + \frac{\ten{F}_{cm}}{M} \; \Delta t,
\end{equation}

\begin{equation}
  \label{eq:rfc:lin_pos_cm_update}
  \ten{x}_{cm}^{n+1} = \ten{x}_{cm}^{n} + \ten{v}_{cm}^{n} \; \Delta t,\\
\end{equation}


The rotation part is updated as


\begin{equation}
  \label{eq:rfc:lin_vel_cm_update}
  \omega^{n+1} = \omega^{n} +  + \frac{T_{cm}}{I_{zz}} \; \Delta t
\end{equation}

\begin{equation}
  \label{eq:rfc:lin_vel_cm_update}
  \ten{\phi}^{n+1} = \ten{\phi}^{n} + \omega \; \Delta t
\end{equation}

Based on the angle, the rotation matrix is updated as

\begin{equation}
  R =
\begin{pmatrix}
    \cos(\phi) & -\sin(\phi) & 0 \\
   \sin(\phi)  & \cos(\phi) & 0 \\
    0 & 0 & 1
\end{pmatrix}
\end{equation}






\FloatBarrier%
\subsection{Freely tranlating and rotating rigid body}
\label{sec:rb_2d}


\FloatBarrier%
\section{Rigid body dynamics 3D}
\label{sec:rb_3d}

% Taken from the appendix of \cite{dietemann2020smoothed},
% \cite{coutsias2004quaternions}, discusses body frame quaternion change
% equation.


\FloatBarrier%
\section{Conclusions}
\label{sec:conclusions}


% \section*{References}


\bibliographystyle{model6-num-names}
\bibliography{references}
\end{document}

% ============================
% Table template for reference
% ============================
% \begin{table}[!ht]
%   \centering
%   \begin{tabular}[!ht]{ll}
%     \toprule
%     Quantity & Values\\
%     \midrule
%     $L$, length of the domain & 1 m \\
%     Time of simulation & 2.5 s \\
%     $c_s$ & 10 m/s \\
%     $\rho_0$, reference density & 1 kg/m\textsuperscript{3} \\
%     Reynolds number & 200 \& 1000 \\
%     Resolution, $L/\Delta x_{\max} : L/\Delta x_{\min}$ & $[100:200]$ \& $[150:300]$\\
%     Smoothing length factor, $h/\Delta x$ & 1.0\\
%     \bottomrule
%   \end{tabular}
%   \caption{Parameters used for the Taylor-Green vortex problem.}%
%   \label{tab:tgv-params}
% \end{table}

%%% Local Variables:
%%% mode: latex
%%% TeX-master: "paper"
%%% fill-column: 78
%%% End:
